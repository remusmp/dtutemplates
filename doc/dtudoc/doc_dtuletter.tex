\documentclass{dtuletter}
\usepackage[utf8]{inputenc}
\usepackage[english]{babel}
\usepackage{verbatim,ifthen,xspace}
\usepackage{array}
\usepackage{longtable}

\newcommand{\optempty}{{\rmfamily\texttt{empty}}}
\newcommand*{\for}[1]{{\rmfamily(#1)}}

% Set department data
\dtuletterconfig{
	%DK-department = DTU ElectrooOOO,
	DK-building = 341,
	DK-street = Elektrovej 3,
	DK-zipcity = 2800 Kgs. Lyngby,
	%
	%UK-department = DTU ElectrooOOO,
	UK-building = 341,
	UK-street = Elektrovej 3,
	UK-zipcity = 2800 Kgs. Lyngby,
	%
	phone=45 32 24 14,
	fax= 45 44 33 22,
	webpage=www.dtu.dk,
	cvr = DK 30 06 09 46
}

% TO data
\dtuletterconfig{
	TO-name = IdentityPeople,
	TO-address = {Vester Farimagsgade 41},
	TO-zipcity = 1608 København V,
	TO-country = Denmark,
	TO-att = Navn Navnesen
}

\dtuletterconfig{
	%stationery = true,
	%lang = english,
	department = business,
	config = user,
	%
	docID = {BG 12-819},
	author = {DJ/hab},
	%
	phone-direct = 45 32 24 11,
	email = tobii@tobii.dk
}

\begin{document}
The DTU letter class is build on the extensive Memoir class by Peter Wilson. The class requires the following packages: placeins, xcolor, calc, graphicx, hyperref, and ifthen. The class uses several options to define the different values of the letter head and foot. All options are activated using the \verb,\dtuletterconfig, command in the preambel, e.g.:

\begin{verbatim}
\dtuletterconfig{
lang = danish,
department = elektro} 
\end{verbatim}

The following table describes the options available. Please note that if you need to write a comma in the text for an option you need to enclose the text in curly braces, ie. \verb.author = {tps, als}..

\begin{longtable}{@{}>{\ttfamily}lp{75mm}>{\ttfamily}l>{}p{30mm}@{}}
\hline
{\normalfont Option} & Description & %{\normalfont Possible Values} 
& {\normalfont Default Value}\\ \hline
lang 		& Sets the language for the letter (department name, footer field names, etc.). (danish, english) & & danish\\
department 	& Specify the department. This selects the correct department logo and text. Possible values are: \verb|aqua|, \verb|byg|, \verb|compute|, \verb|elektro|, \verb|energikonvertering|, \verb|fotonik|, \verb|fysik|, \verb|food|, \verb|kemi|, \verb|kemiteknik|, \verb|management|, \verb|mekanik|, \verb|miljo|, \verb|nanotek|, \verb|space|, \verb|systembiologi|, \verb|transport|, \verb|veterinaerinstituttet|, \verb|vindenergi| & & admin\\
stationery	& If you are printing on stationery paper you should specify this option (true, false) &  & false\\
config 		& Specify a configuration file, ie. specifying \emph{user} will load the file \verb,dtu_letter_user.cfg, if available in the \LaTeX\ search directories or the folder containing the letter. The configuration file could contain the \verb,\dtuletterconfig, command with your default values & & \optempty\\
TO-name 	& The receiver name. & & \optempty\\
TO-address	& The receiver address & & \optempty\\
TO-zipcity	& The receiver zip and city (divided by a space) & & \optempty\\
TO-country	& The receiver country & & \optempty\\
TO-att		& The attention text & & \optempty\\
date 		& The date text. & & \verb,\today,\\
docID  		& Document ID & & \optempty\\
author 		& Document author & &\optempty\\
%
DK-department 	& Danish department name used in the footer (you should not need to set this) & & Set by department  \\
UK-department 	& English department name used in the footer (you should not need to set this) & & Set by department  \\
DK-building 	& Building number for the Danish from address & &\optempty\\
UK-building 	& Building number for the English from address & &\optempty\\
DK-street 		& Street name for the Danish from address & &\optempty\\
UK-street 		& Street name for the English from address & &\optempty\\
DK-zipcity 		& Zip and city for the Danish from address & &\optempty\\
UK-zipcity 		& Zip and city for the English from address & &\optempty\\\\
phone			& Phone number & & \optempty\\
fax				& FAX number. & & \optempty\\
email 			& Email address & & \optempty\\
webpage			& Webpage for the footer (ie. www.dtu.dk) & & \optempty\\
cvr 			& The DTU VAT number & & DK 30 06 09 46\\
phone-direct	& Direct phone number & & \optempty\\
%
%
DK-lang-cvr 			& Word printed before the VAT number (Danish) 			& & CVR-nr.  \\
DK-lang-phone 			& Word printed before the phone number (Danish) 		& & Tlf      \\
DK-lang-phone-direct 	& Word printed before the direct phone number (Danish) 	& & Dir      \\
DK-lang-fax 			& Word printed before the fax number (Danish)			& & Fax      \\
DK-lang-att 			& Word printed before the attention text (Danish)		& & Att.     \\
DK-lang-building 		& Word printed before the building text (Danish)		& & Bygning  \\
UK-lang-dtu				& Offcial DTU name 										& & Danmarks Tekniske Universitet\\
UK-lang-cvr 			& Word printed before the VAT number (English) 			& & VAT no.  \\
UK-lang-phone 			& Word printed before the phone number (English) 		& & Tel      \\
UK-lang-phone-direct 	& Word printed before the direct phone number (English) & & Dir      \\
UK-lang-fax	 			& Word printed before the fax number (English)			& & Fax      \\
UK-lang-att 			& Word printed before the attention text (English)		& & Att.     \\
UK-lang-building  		& Word printed before the building text (English)		& & Building  \\
UK-lang-dtu				& Offcial DTU name 										& & Technical University of Denmark\\
\hline
\end{longtable}

\end{document}